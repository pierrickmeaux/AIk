
\def\anon{0} %% set to 1 for anonymous submissions, hides acknowledgments and author names
\def\full{0} %% set to 0 for springer proceedings
\def\short{0} %% set to 1 for extended abstract version


\documentclass{llncs}

\pagestyle{plain}



\usepackage{makeidx}
\usepackage[dvips]{graphicx}
\usepackage{graphicx}

\usepackage{comment}

\usepackage{listings}
% \usepackage[mathscr]{eucal}
\usepackage{bm}
\usepackage{array}
\usepackage{url}
\usepackage{calc}
\usepackage{float}
\usepackage{latexsym}
% \usepackage{rotating}
\DeclareGraphicsExtensions{.eps,.jpg,.png,.pdf}
\usepackage[usenames, dvipsnames]{xcolor}
\usepackage[sort,nocompress]{cite}
\usepackage{colortbl}
\usepackage{multirow}
\usepackage{lscape}
\usepackage{amsmath}
\let\proof\relax
\let\endproof\relax
\usepackage{amsthm,amsfonts,amssymb}
\usepackage[bookmarks,bookmarksopen,bookmarksdepth=2]{hyperref}
\usepackage{pdflscape}


%\usepackage{natbib}

%  \def\rmdefault{ptm}



\usepackage{setspace}
\usepackage{color}
\ifnum\full=1
\usepackage[margin=0.9in]{geometry}
\usepackage{fullpage}

\setlength{\parskip}{0cm}

%\setstretch{1.03}
%\addtolength{\parskip}{1pt}
% \setcounter{page}{0}
\renewcommand{\tabcolsep}{5pt}
\else
\renewcommand{\tabcolsep}{0pt}
\fi

\renewcommand{\arraystretch}{1.2}

\hyphenpenalty=5000
\tolerance=1000




%\ifnum\full=1
%\usepackage{natbib}
%\bibliographystyle{alpha}
%\setlength{\bibsep}{0pt}
%\renewcommand{\bibsection}{\section*{References}\small}
%\else
%\usepackage[numbers]{natbib}
%\bibliographystyle{splncs04}
%\fi



\DeclareMathOperator{\Exp}{E}
\DeclareMathOperator{\Var}{Var}
\DeclareMathOperator{\poly}{poly}
\DeclareMathOperator{\Supp}{Supp}

\usepackage{enumitem}

\usepackage{tikz}
\usetikzlibrary{arrows,shapes}
\usetikzlibrary{plotmarks}


%notes

%\definecolor{myorange}{rgb}{0.99,0.6,0.25}
%\newcommand{\pmnote}[1]{\colorbox{myorange}{\parbox{0.9\linewidth}{[{\footnotesize {\bf PM:} { {#1}}}]}}}


\definecolor{mycolor}{rgb}{0.99,0.33,0.44}
\newcommand{\pmnote}[1]{\colorbox{mycolor}{\parbox{0.9\linewidth}{[{\footnotesize {\bf PM:} { {#1}}}]}}}

\definecolor{unmellowyellow}{rgb}{1.0, 1.0, 0.4}
\newcommand{\agnote}[1]{\colorbox{unmellowyellow}{\parbox{0.9\linewidth}{[{\footnotesize {\bf AG:} { {#1}}}]}}}
%% Sets

\newcommand{\Z}{\mathbb{Z}}
\newcommand{\N}{\mathbb{N}}
\newcommand{\R}{\mathbb{R}}
\newcommand{\F}{\mathbb{F}}
\newcommand{\Znm}{\mathbb{Z}_q^{n \times m}}

%matrices
\newcommand{\matzero}{\mathbf{0}}
\newcommand{\matA}{\mathbf{A}}
\newcommand{\matB}{\mathbf{B}}
\newcommand{\matC}{\mathbf{C}}
\newcommand{\matE}{\mathbf{E}}
\newcommand{\matF}{\mathbf{F}}
\newcommand{\matG}{\mathbf{G}}
\newcommand{\matI}{\mathbf{I}}
\newcommand{\matM}{\mathbf{M}}
\newcommand{\matP}{\mathbf{P}}
\newcommand{\matR}{\mathbf{R}}
\newcommand{\matS}{\mathbf{S}}
\newcommand{\matT}{\mathbf{T}}
\newcommand{\matU}{\mathbf{U}}
\newcommand{\matV}{\mathbf{V}}
\newcommand{\matW}{\mathbf{W}}
\newcommand{\matX}{\mathbf{X}}
\newcommand{\matY}{\mathbf{Y}}
\newcommand{\matZ}{\mathbf{Z}}


%vectors
\newcommand{\veca}{\mathbf{a}}
\newcommand{\vecb}{\mathbf{b}}
\newcommand{\vecc}{\mathbf{c}}
\newcommand{\vecd}{\mathbf{d}}
\newcommand{\vece}{\mathbf{e}}
\newcommand{\veci}{\mathbf{i}}
\newcommand{\vecj}{\mathbf{j}}
\newcommand{\veck}{\mathbf{k}}
\newcommand{\vecl}{\mathbf{l}}
\newcommand{\vecm}{\mathbf{m}}
\newcommand{\vecp}{\mathbf{p}}
\newcommand{\vecr}{\mathbf{r}}
\newcommand{\vecs}{\mathbf{s}}
\newcommand{\vecv}{\mathbf{v}}
\newcommand{\vecw}{\mathbf{w}}
\newcommand{\vecu}{\mathbf{u}}
\newcommand{\vecx}{\mathbf{x}}
\newcommand{\vecy}{\mathbf{y}}
\newcommand{\vecz}{\mathbf{z}}





%FiLIP notations

\newcommand{\FLIP}{\textsf{FLIP}}
\newcommand{\IFPl}{\text{Improved Filter Permutator} }
\newcommand{\IFPs}{\text{IFP} }

\newcommand{\FiLIP}{\textsf{FiLIP}}
\newcommand{\FiLIPDSM}{\mathsf{FiLIP}_{\mathsf{DSM}}}
\newcommand{\FiLIPXMAJ}{\mathsf{FiLIP}_{\mathsf{XMAJ}}}

%Boolean functions

\newcommand{\Bfn}[1]{\mathcal{B}_{#1}}
\newcommand{\BN}{\mathcal{B}_n}
\newcommand{\Bn}[1]{\mathcal{B}_{#1}}
\newcommand{\Bnstar}[1]{\mathcal{B}_{#1}^*}

\newcommand{\Bvad}[3]{\mathcal{B}({#1},{#2},{#3})}


\newcommand{\AI}{\mathsf{AI}}
\newcommand{\AN}{\mathsf{AN}}
%\newcommand{\difAN}[1]{\Delta_{\mathsf{AN}}(#1)}
%\newcommand{\DAN}{\mathsf{d}\mathsf{AN}}
%\newcommand{\Sd}{\mathsf{S}_\mathsf{d}}
\newcommand{\SD}{\mathsf{SD}}
\newcommand{\FAI}{\mathsf{FAI}}
\newcommand{\NL}{\mathsf{NL}}
\newcommand{\NLk}[1]{\mathsf{NL}_{#1}}
\newcommand{\AIk}[1]{\mathsf{AI}_{#1}}
%\newcommand{\NLd}{\mathsf{NL_d}}
\newcommand{\res}{\mathsf{res}}
\newcommand{\bal}{\mathsf{bal}}
\newcommand{\gnlk}{\mathsf{GWNL}}


\newcommand{\NPB}{\mathsf{NPB}}

\newcommand{\DS}[1]{\mathsf{DS}(#1)}
\newcommand{\DSR}[2]{\mathsf{DS}^{#2}(#1)}
%\newcommand{\matAI}[3]{\mathbf{A}_{#2,#3}(#1)}

\newcommand{\WPB}[1]{\mathcal{WPB}_{#1}}
\newcommand{\WAPB}[1]{\mathcal{WAPB}_{#1}}
\newcommand{\SWAPB}[1]{\mathcal{SWAPB}_{#1}}
\newcommand{\SYM}[1]{\mathcal{SYM}_{#1}}
%for affine weightwise: degree and number of variables
\newcommand{\WD}[2]{\mathcal{WD}^{#1}_{#2}}
\newcommand{\Ekn}[2]{\mathsf{E}_{#1,#2}}
\newcommand{\Code}[2]{\mathsf{P}_{#1,#2}}
\newcommand{\mdist}[2]{\mathsf{d}_{#1,#2}}



\newcommand{\mnlk}[2]{\mu_{#1,#2}}
\newcommand{\Mnlk}[2]{\mathsf{M}_{#1,#2}}
\newcommand{\mnl}[1]{\mu_{#1}}
\newcommand{\Mnl}[1]{\mathsf{M}_{#1}}

\newcommand{\DistWkn}[2]{\mathfrak{W}_{#1,#2}}
\newcommand{\DistWn}[1]{\mathfrak{W}_{#1}}
\newcommand{\Dkn}[2]{\mathfrak{D}_{#1,#2}}
\newcommand{\Dn}[1]{\mathfrak{D}_{#1}}

\newcommand{\kraw}[3]{\mathsf{K}_{#1}(#2,#3)}
\newcommand{\phikn}[2]{\varphi_{#1,#2}}

\newcommand{\const}[2]{g_{#1,#2}}
\newcommand{\setn}[1]{S_{#1}}
\newcommand{\symsetsmall}[1]{A_{#1}}
\newcommand{\symset}[2]{B_{#1,#2}}


%usual notations
\newcommand{\supp}{\mathsf{supp}}
\newcommand{\suppk}[1]{\mathsf{supp}_{#1}}
\newcommand{\w}{\mathsf{w_H}}
\newcommand{\hd}{\mathsf{d_H}}
\newcommand{\degg}{\mathsf{deg}}
\newcommand{\Span}{\mathsf{Span}}
\newcommand{\rank}{\mathsf{rank}}
%Walsh transform
\newcommand{\wt}[1]{W_{#1}} 
\newcommand{\Wsupp}[1]{\mathsf{Wsupp}_{#1}} 
%restricted Walsh transform W_k,a (f)
\newcommand{\wtk}[2]{\mathcal{W}_{#1,#2}} 

%S-equivalent classes
\newcommand{\sclass}[1]{\mathcal{S}(#1)}
\renewcommand{\S}{\mathcal{S}}
\newcommand{\Sz}{\mathcal{S}_0}
\newcommand{\sclassz}[1]{\mathcal{S}_0(#1)}
\newcommand{\degsy}{ds}
%\newcommand{\degsz}{\degg_{\Sz}}
\newcommand{\degsz}{\mathsf{\sigma deg}}

\newcommand{\minAI}[1]{\mathsf{mAI}\mathcal{S}_0(#1)}
\newcommand{\minNL}[1]{\mathsf{mNL}\mathcal{S}_0(#1)}
\newcommand{\mindegg}[1]{\mathsf{mdeg}\mathcal{S}_0(#1)}

\newcommand{\set}[1]{\left\{#1\right\}}
\newcommand{\mAN}[2]{\mathsf{d}_{#1}^{#2}}


%gates
\newcommand{\AND}{\textsf{AND}}
\newcommand{\XOR}{\textsf{XOR}}
\newcommand{\MUX}{\textsf{MUX}}


%families of functions
\newcommand{\MAJ}{\textsf{MAJ}}
\newcommand{\DSM}{\textsf{DSM}}
\newcommand{\XORTHR}{\textsf{XOR-THR}}
\newcommand{\XORMAJ}{\textsf{XOR-MAJ}}

\newcommand{\xorlk}[2]{{\mathsf{XOR}}_{#1}  \mathsf{M}_{#2}} 
\newcommand{\xormaj}[2]{{\mathsf{XOR}}_{#1}  \mathsf{MAJ}_{#2}} 
%\newcommand{\xorthr}[3]{{\mathsf{XOR}}_{#1}  \mathsf{T}_{{#2},{#3}}} 
\newcommand{\xorthr}[3]{{\mathsf{XOR}}_{#1}+\mathsf{T}_{{#2},{#3}}}
\newcommand{\tri}[1]{{T}_{#1}}
\newcommand{\thr}[2]{\mathsf{T}_{{#1},{#2}}}
\newcommand{\xor}[1]{\mathsf{XOR}_{#1}}
\newcommand{\maj}[1]{\mathsf{MAJ}_{#1}}


\newcommand{\nbf}[1]{\mathsf{C}_{#1}}
\newcommand{\nbfodd}[2]{\mathsf{A}_{#1,#2}}
\newcommand{\nbfeven}[2]{\mathsf{B}_{#1,#2}}

%direct sum vector and simplified value vector
\newcommand{\dsv}[1]{\mathbf{m}_{#1}}
\newcommand{\svv}[1]{\mathbf{s}_{#1}}



\newtheorem{Prop}{Property}
\newtheorem{Cons}{Construction}

\newtheorem{tables}{Tables}

% For algorithms
\usepackage{algorithm,algpseudocode}

\renewcommand{\algorithmicrequire}{\textbf{Input:}}
\renewcommand{\algorithmicensure}{\textbf{Output:}}
% \renewcommand{\ALG@name}{Construction}
\newenvironment{constr}[1][htb]{%
\floatname{algorithm}{Construction}% Update algorithm name
   \begin{algorithm}[#1]%
  }{\end{algorithm}}
 

\algnewcommand\algorithmicparfor{\textbf{par-for}}
\algdef{S}[FOR]{ParFor}[1]{\algorithmicparfor\ #1\ \algorithmicdo}
 
%latin

\newcommand{\ie}{\textit{i.e.} }
\newcommand{\eg}{\textit{e.g.} }
\newcommand{\ea}{\textit{et al.} }




\begin{document}
%	\title{Actions of symmetric functions over weightwise almost perfectly balanced functions}
	\title{On the restricted algebraic immunity of weightwise almost perbeftly balanced functions}

\ifnum\anon=0
\author{
	\mbox{Agnese Gini\orcidID{0009-0001-9565-380X}, Pierrick M\'eaux\orcidID{0000-0001-5733-4341}}%\inst{1}
}
\authorrunning{ A. Gini, P. M\'eaux}
\institute{
	University of Luxembourg, Luxembourg\\
	\email{agnese.gini@uni.lu, pierrick.meaux@uni.lu}		
}
\fi
	
	
	
	%----------------------------------------------------------------
	\maketitle
	\author{\mbox{Agnese Gini, Pierrick M\'eaux}%\inst{1}
}

\authorrunning{ A. Gini, P. M\'eaux}

\institute{
	University of Luxembourg, Luxembourg\\
	\email{agnese.gini@uni.lu, pierrick.meaux@uni.lu}		
}	

\setcounter{page}{1}	

\begin{abstract}
 
	\end{abstract}
	
	
\section{Introduction}

	
%	Weightwise perfectly balanced functions (WPB) 
	
%WPB constructions:
%\cite{TOSC:CarMeaRot17}
%\cite{DCC:LiuMes19,CC:TangLiu19,DAM:LiSu20,CC:MesSu21,AMC:ZhangSu21,BFA:MesSuLi21,DAM:GuoSu22,DAM:ZhuSu22,CEC:MPJDL22,DAM:GinMea22,INDO:GinMea22,ACI:ManKuLar22,CC:MSLZ22,eprint:GinMea22c,Mat:ZJZQ23,eprint:ZLCQZ23,eprint:GinMea23a}.


%Potential road map:
%\begin{itemize}
%	\item S-equivalent class on WPB functions, definition, and properties: 
%	
%It maintains the AIk, NLk and NPBness. 
%
%The degree can vary + result
%
%The NL and AI can vary, we have it experimentally, but we can also prove it.
%	\item Experimental 
%
%Variation of AI and NL on some known classes
%
%	
%\item Build optimal function in the class?	
%
%experimentally, give it from some known functions
%
%theoretically, if possible, give a construction that optimizes the (NL, deg and AI) parameters inside a class.
%	
%	\item Definition for WAPB functions, + extension of the results.
%\end{itemize}






\section{Preliminaries}

\ifnum\short=0
For readability we use the notation $+$ instead of 
$\oplus$ to denote the addition in $\F_2$ and $\sum$ instead of $\bigoplus$. 
In addition to classic notations we denote by $ [a,b] $ the subset of all integers between $a$ and $b$: $\{a, a+1, \ldots,b\}$. 

For a vector $v\in \F_2^n$ we use $\w(v)$ to denote its Hamming weight $\w(v)=|\{ i \in [1,n] \, | \, v_i=1 \}|$. For two vectors $v$ and $w$ of $\F_2^n$ we denote $\hd(v,w)$ the Hamming distance between $v$ and $w$, that is $\hd(v,w)=\w(v+w)$.
\fi
\ifnum\short=0
\subsection{Boolean functions and weightwise considerations}

In this part we recall general concepts on Boolean functions and their weightwise properties we use in this article. 
For a deeper introduction on Boolean functions and their cryptographic parameters we refer to \eg the book~\cite{Carlet20} and to~\cite{TOSC:CarMeaRot17} for the weightwise properties, also called properties on the slices.

For $k \in [0,n]$ we denote $\Ekn{k}{n}$ the set $\{x\in \F_2^n \, | \, \w(x)=k  \}$ and call it slice of the Boolean hypercube (of dimension $n$). 
Accordingly, the Boolean hypercube is partitioned into $n+1$ slices where the elements have the same Hamming weight.


\begin{definition}[Boolean Function]\label{def:bool_f}
	A Boolean function $f$ in $n$ variables is a function from $\F_2^n$ to $\F_2$. 
	The set of all Boolean functions in $n$ variables is denoted by $\BN$, and we denote $\BN^*$ the set without the null function.
\end{definition}
\else 

A \emph{Boolean function} $f$ in $n$ variables is a function from $\F_2^n$ to $\F_2$. 
We recall here general concepts on Boolean functions and their weightwise properties, we refer to \eg~\cite{Carlet20} and to~\cite{TOSC:CarMeaRot17} respectively for further details.
	The set of all Boolean functions in $n$ variables is denoted by $\BN$, and we denote $\BN^*$ the set without the null function. We call \emph{Algebraic Normal Form} of a Boolean $n$-variable polynomial representation over $\F_2$ (\textit{i.e.} in $\F_2[x_1,\dots,x_n]/(x_1^2+x_1,\dots,x_n^2+x_n)$):
$f(x_1,\dots,x_n)= \sum_{I \subseteq [1,n]} a_I \left( \prod_{i \in I} x_i \right)$ where $a_I\in \F_2$. 
The \emph{(algebraic) degree} of $f$, denoted $\degg(f)$ is $\degg(f)=\max_{I\subseteq [1,n]}\{ |I|\, | \, a_I=1\}  \text{ if $f$ is not null},0  \text{ otherwise}.$

\fi	

\ifnum\short=1
% \vspace{-10pt}
% \subsubsection{Restricted properties.}
\fi 

To denote when a property or a definition is restricted to a slice we use the subscript $k$. 
For example, for a $n$-variable Boolean function $f$ we denote its support $\supp(f)=\{x\in \F_2^n \, | \, f(x)=1  \}$ and we denote $\suppk{k}(f)$ its support restricted to a slice, that is $\supp(f)\cap \Ekn{k}{n}$.

\ifnum\short=1
A Boolean function $f\in \BN$ is called \emph{balanced} if $|\supp(f)|=2^{n-1}=|\supp(f+1)|$.	For $k\in [0,n]$ the function is said \emph{balanced on the slice} $k$ if $||\suppk{k}(f)|-|\suppk{k}(f+1)| |\le 1$. In particular when $|\Ekn{k}{n}|$ is even $|\suppk{k}(f)|=|\suppk{k}(f+1)|=|\Ekn{k}{n}|/2$. 

Let $m\in \N^*$ and $n=2^m$, $f$ is called \emph{weightwise perfectly balanced} (WPB) if, for every $k\in[1,n-1]$, $f$ is balanced on the slice $k$, that is $\forall k \in [1,n-1], |\suppk{k}(f)|=\binom{n}{k}/2$, and $f(0_n)=0$ and $f(1_n)=1$.	The set of WPB functions in $2^m$ variables is denoted $\WPB{m}$. When $n$ is not a power of $2$, other weights than $k=0$ and $n$ give slices of odd cardinality, in this case we call $f\in \Bn{n}$ \emph{weightwise almost perfectly balanced} (WAPB)  if $|\suppk{k}(f)|$ is either $|\Ekn{k}{n}|/2$  if $|\Ekn{k}{n}|$  is even, or $(|\Ekn{k}{n}|\pm 1)/2$, otherwise.
	The set of WAPB functions in $n$ variables is denoted $\WAPB{n}$.
\else
\begin{definition}[Balancedness]\label{def:balancedness}
	A Boolean function $f\in \BN$ is called balanced if $|\supp(f)|=2^{n-1}=|\supp(f+1)|$. 
	For $k\in [0,n]$ the function is said balanced on the slice $k$ if $||\suppk{k}(f)|-|\suppk{k}(f+1)| |\le 1$. In particular when $|\Ekn{k}{n}|$ is even $|\suppk{k}(f)|=|\suppk{k}(f+1)|=|\Ekn{k}{n}|/2$.
\end{definition}


\begin{definition}[Weightwise (Almost) Perfectly Balanced Function (WPB and WAPB)]\label{def:WPB}
	Let $m\in \N^*$ and $f$ be a Boolean function in $n=2^m$ variables. It will be called weightwise perfectly balanced (WPB) if, for every $k\in[1,n-1]$, $f$ is balanced on the slice $k$, that is $\forall k \in [1,n-1], |\suppk{k}(f)|=\binom{n}{k}/2$, and:
	\ifnum\short=1
	\vspace{-0.3cm}
	\else
	\fi
	\[f(0,\cdots,0)=0,\quad \text{ and } f(1,\cdots,1)=1.\]	
	The set of WPB functions in $2^m$ variables is denoted $\WPB{m}$.
	
	When $n$ is not a power of $2$, other weights than $k=0$ and $n$ give slices of odd cardinality, in this case we call $f\in \Bn{n}$ weightwise almost perfectly balanced (WAPB) if: 
	\[|\suppk{k}(f)|= \left \{
	\begin{array}{l l}
	|\Ekn{k}{n}|/2  & \text{ if } |\Ekn{k}{n}| \text{ is even, } \\
	(|\Ekn{k}{n}|\pm 1)/2  & \text{ if }  |\Ekn{k}{n}| \text{ is odd.}
	\end{array}\right.\]
	The set of WAPB functions in $n$ variables is denoted $\WAPB{n}$.	
	
\end{definition}
\fi
% \ifnum\short=0
% The first WAPB family of function has been exhibited in~\cite{TOSC:CarMeaRot17}:
% \begin{definition}[CMR WAPB construction (adapted from~\cite{TOSC:CarMeaRot17}, Proposition $5$)]\label{def:CMR}
% 	Let $n \in \N, n\ge 2$, the WAPB function $f_n$ is recursively defined by $f_2(x_1,x_2)=x_1$ and for $n\geq 3$:
% 	\[f_n(x_1,\dots ,x_n)= \left \{
% 	\begin{array}{l l}
% 	f_{n-1}(x_1,\dots ,x_{n-1})   & \text{ if } n \text{ odd, } \\
% 	f_{n-1}(x_1,\dots ,x_{n-1})+x_{n-2}+  \prod_{i=1}^{2^{d-1}} x_{n-i}  & \text{ if } n=2^d; d>1, \\
% 	f_{n-1}(x_1,\dots ,x_{n-1})+x_{n-2}+  \prod_{i=1}^{2^d} x_{n-i} & \text{ if } n=p\cdot2^d; p\text{ odd}.
% 	\end{array}
% 	\right .\]
% 	
% %	Re-indexing the variables the subfamily of WPB functions (the cases where $n$ is a power of $2$) can be written as:
% %	\[f(x_1, x_2, \dots, x_{2^m})=\sum_{a=1}^{m} \sum_{i=1}^{2^{m-a}} \prod_{j=0}^{2^{a-1}-1} x_{i+j2^{m-a+1}}\]	
% \end{definition}
% \else
% The first WAPB family of function has been exhibited in~\cite[Proposition 5]{TOSC:CarMeaRot17} and it is usually referred as CMR functions. 
% \fi




\ifnum\short=1
\else

\begin{definition}[Walsh transform and restricted Walsh transform]\label{def:walsh_transform}
	Let $f\in \Bn{n}$ be a Boolean function, its Walsh transform $\wt{f}$ at $a \in \F_2^n$ is defined as:
	\[  \wt{f} (a) := \sum_{x \in \F_2^n} (-1)^{f(x) +  a \cdot x }.\]
	%	The Walsh support is the set $\Wsupp{f}:=\{ a\in \F_2^n\, | \, \wt{f} (a) \neq 0 \}$.
	Let $f\in \Bn{n}$, $S \subset \F_2^n$, its Walsh transform restricted to $S$ at $a \in \F_2^n$ is defined as:
	\[  \wt{f,S} (a) := \sum_{x\in S} (-1)^{f(x)+ax}.\]
	For $S=\Ekn{k}{n}$ we denote $\wt{f,\Ekn{k}{n}} (a)$ by $\wtk{f}{k}(a)$, and for $a= 0_{n}$ we denote $\wtk{f}{k}(a)$ as $\wtk{f}{k}(0)$.
\end{definition}
\fi


\ifnum\short=1

The \emph{nonlinearity} $\NL(f)$ of $f\in \BN$ is the minimum Hamming distance between $f$ and all the affine functions in $\BN$, \ie $ \NL(f) = \min_{g,\, \degg(g)\leq 1} \{ \hd(f,g) \}$.
	For $k\in [0,n]$ we denote $\NLk{k}$ the \emph{nonlinearity on the slice} $k$, the minimum Hamming distance between $f$ restricted to $\Ekn{k}{n}$ and the restrictions to $\Ekn{k}{n}$ of affine functions over $\F_2^n$, \ie $ \NLk{k}(f)=\min_{g,\, \degg(g)\leq 1} |\suppk{k}(f+g)|$.
\else
\begin{definition}[Nonlinearity and weightwise nonlinearity] \label{def:nl}
	The nonlinearity $\NL(f)$ of a Boolean function $f\in \BN$, where $n$ is a positive integer, is the minimum Hamming distance between $f$ and all the affine functions in $\BN$:
	\ifnum\short=1
	\vspace{-0.3cm}
	\else
	\fi
	\[ \NL(f) = \min_{g,\, \degg(g)\leq 1} \{ \hd(f,g) \}{,} \]
	where $g(x)=a\cdot x+\varepsilon$, $a\in \F_2^n, \varepsilon\in \F_2$ 
	(where $\cdot$ is an inner product in $\F_2^n$, any choice of inner product will give the same value of $\NL(f)$)
	
	For $k\in [0,n]$ we denote $\NLk{k}$ the nonlinearity on the slice $k$, the minimum Hamming distance between $f$ restricted to $\Ekn{k}{n}$ and the restrictions to $\Ekn{k}{n}$ of affine functions over $\F_2^n$. 
	Accordingly: 
	\ifnum\short=1
	\vspace{-0.3cm}
	\else
	\fi
	\[\NLk{k}(f)=\min_{g,\, \degg(g)\leq 1} |\suppk{k}(f+g)|.\]
%	We refer to the global weightwise nonlinearity of $f$ as $\gnlk(f)=\sum_{k=0}^{n}\NLk{k}(f)$.
\end{definition}
\fi

\ifnum\short=1
\else
\begin{Prop}[Nonlinearity on the slice, adapted from~\cite{TOSC:CarMeaRot17}, Proposition $6$]\label{prop:nlk}
	Let $n\in \N^*, k\in [0,n]$, for every $n$-variable Boolean function $f$ over $\Ekn{k}{n}$:
	\[ \NLk{k}(f)=\frac{|\Ekn{k}{n}|}{2} -\frac{\max_{a\in \F_2^n} |\wtk{f}{k}(a)|}{2}  .\]
\end{Prop}


\begin{definition}[Non Perfect Balancedness (\cite{C2SI:GM})]
	Let $m\in\N^*$, $n=2^m$, and $f$ an $n$-variable Boolean function, the non perfect balancedness of $f$, denoted by $\NPB(f)$ is defined as:
	\[\NPB(f)=\min_{g \in \WPB{m}} \hd(f,g).\]
\end{definition}

\begin{Prop}[NPB and restricted Walsh transform (~\cite{C2SI:GM}, Proposition $2$)]\label{prop:NPBrestWalsh}
	Let $m\in\N^*$, $n=2^m$, and $f\in \Bfn{n}$, the following holds on its non perfect balancedness:
	\[\displaystyle \NPB(f)= \frac{2-\wtk{f}{0}(0)+ \wtk{f}{n}(0)}{2} + \sum_{k=1}^{n-1}\frac{ |\wtk{f}{k}(0)|}{2} .\]
\end{Prop}
\fi


\ifnum\short=0
\begin{definition}[Algebraic Normal Form (ANF) and degree]\label{def:anf}
We call Algebraic Normal Form of a Boolean function $f$ its $n$-variable polynomial representation over $\F_2$ (\textit{i.e.} belonging to $\F_2[x_1,\dots,x_n]/(x_1^2+x_1,\dots,x_n^2+x_n)$):
\ifnum\short=1
\vspace{-0.3cm}
\else
\fi
	\[f(x_1,\dots,x_n)= \sum_{I \subseteq [1,n]} a_I \left( \prod_{i \in I} x_i \right) \]%x_1,\dots,x_n=\sum_{I \subseteq [1,n]} a_I x^I,\]
	where $a_I\in \F_2$. 
%The (algebraic) degree of $f$, denoted $\degg(f)$ is: \[\degg(f):=\left\{\begin{array}{l}
%	\max_{I\subseteq [1,n]}\{ |I|\, | \, a_I=1\}  \text{ if $f$ is not null}\\
%	0  \text{ otherwise}.
%	\end{array}\right.\]	
The (algebraic) degree of $f$, denoted $\degg(f)$ is:
\ifnum\short=1
\vspace{-0.3cm}
\else
\fi
 \[\degg(f)=\
\max_{I\subseteq [1,n]}\{ |I|\, | \, a_I=1\}  \text{ if $f$ is not null},0  \text{ otherwise}.\]
\end{definition}	
\fi

%\begin{Prop}[Minimum degree of WPB function, adapted from~\cite{TOSC:CarMeaRot17}, Proposition $4$]\label{prop:WPBdeg}
%	Let $m\in \N^*$ for every $2^m$-variable WPB function $f$, the following holds: $\degg(f)\ge 2^{m-1}$.
%\end{Prop}

\ifnum\short=1

The \emph{algebraic immunity} (AI) of a Boolean function $f\in \Bfn{n}$, denoted as $\AI(f)$, is defined as:
	$ \AI(f) = \min_{g \neq 0}\{ \degg(g) \; | \; fg = 0 \; \text{or} \; (f + 1)g = 0 \}.$
The function $g$ is called an \emph{annihilator} of $f$ (or $f + 1$). The \emph{weightwise algebraic immunity} on the slice $\Ekn{k}{n}$, denoted by $\AIk{k}(f)$, is defined as: $\displaystyle \min\set{\degg(g) \; | \; fg = 0 \; \text{or} \; (f + 1)g = 0 \text{ over } \Ekn{k}{n}}$ where $g$ is non null on $\Ekn{k}{n}$.
\else
\begin{definition}[Algebraic Immunity (AI), and weightwise AI] \label{def:ai}
The Algebraic Immunity (AI) of a Boolean function $f\in \Bfn{n}$, denoted as $\AI(f)$, is defined as:
\ifnum\short=1
	$ \displaystyle \AI(f) = \min_{g \neq 0}\{ \degg(g) \; | \; fg = 0 \; \text{or} \; (f + 1)g = 0 \}{,}$
\else
\[ \AI(f) = \min_{g \neq 0}\{ \degg(g) \; | \; fg = 0 \; \text{or} \; (f + 1)g = 0 \}{,} \]
\fi
	where $\degg(g)$ is the algebraic degree of $g$.	
	The function $g$ is called an annihilator of $f$ (or $f + 1$). 
\ifnum\short=1
\else
	Additionally we denote $\AN(f) = \min_{g \neq 0}\{ \degg(g) \; | \; fg = 0\}$.
\fi
	
 	The weightwise algebraic immunity of a Boolean function $f\in \Bfn{n}$ on the slice $\Ekn{k}{n}$, denoted as $\AIk{k}(f)$, is defined as: 
 	$\min\set{\degg(g) \; | \; (fg)= 0 \; \text{or} \; (f + 1)g = 0 \text{ over } \Ekn{k}{n}}$ where $g$ is non null on $\Ekn{k}{n}$.
\end{definition}
\fi


\ifnum\short=1
\else
\begin{Prop}\label{prop:suppannhil}
If $g\in \Bfn{n}^*$ is an annihilator of $f$ and $h$ another function such that $\supp(h)\subseteq \supp(g)$, then $hf=0$. 
\end{Prop}
\fi

% \begin{definition}[Algebraic Immunity over the slices]\label{def:aik}
% 	The algebraic immunity of a Boolean function $f\in \Bfn{n}$ over the slice $\Ekn k n $, denoted by $\AI_k(f)$, is defined as:
% 	\[ \AI_k(f) = \min_{g\in \Bfn{n}}\{ \degg(g) \; | \; (fg)_{\mid_{\Ekn k n}} = 0 \; \text{or} \; ((f + 1)g)_{\mid_{\Ekn k n}} = 0 \;  \text{and} \; g_{\mid_{\Ekn k n}} \neq  0 \}{.} \]
% \end{definition}

\ifnum\short=1
\else


\subsection{Families of WPB functions}
In this section we recall families of WPB functions exhibited in former works. 
These families will be used as examples of WPB functions with minimal or maximal parameters relatively to the degree or algebraic immunity.


\begin{definition}[CMR WAPB construction (adapted from~\cite{TOSC:CarMeaRot17}, Proposition $5$)]\label{def:CMR}
	Let $n \in \N, n\ge 2$, the WAPB function $f_n$ is recursively defined by $f_2(x_1,x_2)=x_1$ and for $n\geq 3$:
	\[f_n(x_1,\dots ,x_n)= \left \{
	\begin{array}{l l}
	f_{n-1}(x_1,\dots ,x_{n-1})   & \text{ if } n \text{ odd, } \\
	f_{n-1}(x_1,\dots ,x_{n-1})+x_{n-2}+  \prod_{i=1}^{2^{d-1}} x_{n-i}  & \text{ if } n=2^d; d>1, \\
	f_{n-1}(x_1,\dots ,x_{n-1})+x_{n-2}+  \prod_{i=1}^{2^d} x_{n-i} & \text{ if } n=p\cdot2^d; p\text{ odd}.
	\end{array}
	\right .\]
	
	Re-indexing the variables the subfamily of WPB functions (the cases where $n$ is a power of $2$) can be written as:
	\[f(x_1, x_2, \dots, x_{2^m})=\sum_{a=1}^{m} \sum_{i=1}^{2^{m-a}} \prod_{j=0}^{2^{a-1}-1} x_{i+j2^{m-a+1}}\]
	%is a weightwise almost perfectly balanced Boolean function of degree $2^{d-1}$, where $2^d\leq n < 2^{d+1}$, and with $n-1$ monomials in its ANF if $n$ is even and $n-2$ monomials if $n$ is odd. Note that this function can be written as a direct sum for all $n\geq 2$.	
\end{definition}


%For example, the $16$-variable function from this construction is:
%\begin{multline*}
%f_{16}=x_1+x_2+x_2 x_3 + x_4 +x_4 x_5 +x_6+ x_4 x_5 x_6 x_7  \\ + x_8 +x_8 x_9 + x_{10} +x_8 x_9 x_{10} x_{11} +x_{12}+x_{12}x_{13}+ x_{14}+x_8 x_9 x_{10} x_{11} x_{12} x_{13} x_{14} x_{15}.
%\end{multline*}


%\begin{Prop}[CMR WPB functions properties~\cite{TOSC:CarMeaRot17}]\label{Prop:CMR}
%	Let $m \in \N^*$ and $n=2^m$, the $n$-variable CMR function $f_n$ has the following properties:
%	
%	$\degg(f_n)=\frac{n}{2}$.
%\end{Prop}



\begin{definition}[LM WPB construction (adapted from~\cite{DCC:LiuMes19}, Corollary $3.5$)]\label{def:LM}
	Let $n \in \N, n\ge 2$, we denote by $\Gamma_n$ the set of all the coset leaders of the cyclotomic classes  of $2$ modulo $2^n- 1$ and by  $o(j)$  the cardinality of the cyclotomic class of $2$ modulo $2^n- 1$ containing $j$.
	Define $T_j\colon \F_{2^{o(j)}}\rightarrow \F_{2}$ the function $y \mapsto \sum_{i=0}^{o(j)-1} y^{2^i}$. 
	For any fixed $\beta$ primitive element of $\F_{2^2}$ and any given any function $\iota\colon \Gamma_n\setminus\set{0} \rightarrow \set{1,2} $, the LM WPB function associate to $\iota$ is   
	\[
	g_\iota (x) = \sum_{j\in\Gamma_n\setminus\set{0} } T_j(\beta^{\iota(j)} x^j).
	\]
\end{definition}

%\begin{Prop}[LM WPB functions properties~\cite{DCC:LiuMes19}]\label{Prop:LM}
%	Let $m \in \N^*$ and $n=2^m$, a $n$-variable LM function $g_n$ has degree 	$\degg(g_n)=n-1$.
%\end{Prop}

\begin{definition}[TL WPB construction (adapted from~\cite{CC:TangLiu19}, Construction $1$ )]\label{def:TL}
	Let $m \in \N^*$ and $n = 2^m \ge 4$ be an integer. A TL WPB Boolean function $h$ on $n$-variable is such that
	\begin{itemize}
	 \item $h(0_n)=0$ and $h(1_n)=1$ 
	 \item $h(x,y)=0$ if $\w(x)<\w(y)$, where $x,y \in \F_2^{m-1}$.
	 \item $h(x,y)=1$ if $\w(x)>\w(y)$, where $x,y \in \F_2^{m-1}$.
	 \item the cardinality of $U_i=\supp(f)\cap\set{(x,y)\in\F_2^{2^{m-1}}\times \F_2^{2^{m-1}}\colon \w(x)=\w(y)=i}$ is exactly $\binom{2^{m-1}}{j}^2/2$ for all $0<j<2^{m-1}$.
	\end{itemize}	
	% 	Let $(c_0,\dots,c_k)\in \set{0,1}^{k+1}$ such that $\sum c_j=0$
% 	\[
% 	 c_j=
% 	 \begin{cases}
% 	  0 & |\Ekn{j}{k}|\equiv 0 \pmod 2\\ 
% 	  \pm 1 & |\Ekn{j}{k}|\equiv 1 \pmod 2\\
% 	 \end{cases}
% 	\]
\end{definition}
\begin{remark}
 Despite Definition~\ref{def:TL} may appear quite different from the original paper, it is equivalent when applying the constrains from the definitions we consider. 
 Namely, here we consider only the case where $n$ is a power of two. Referring to Construction $1$ of~\cite{CC:TangLiu19}, this implies that the coefficients $c_1,\dots,c_{k-1}$ must be zero. Moreover, in~\cite{CC:TangLiu19} $f(0_n)=0$ and $f(1_n)=1$ is not required for weightwise perfectly balancedness, differently from Definition~\ref{def:WPB}. 
 This implies that in this context we can only instantiate the construction with $(-1,0,\ldots,0,1)$ as input sequence, \ie as in Definition~\ref{def:TL}. 
\end{remark}

%\begin{Prop}[TL WPB functions properties~\cite{CC:TangLiu19}]\label{Prop:TL}
%	Let $m \in \N^*$ and $n=2^m$, a  $n$-variable TL function $h_n$ has optimal algebraic immunity $\AI(h_n)=\frac{n}{2}$.
%\end{Prop}


\begin{Prop}[Properties of WPB families, ~\cite{TOSC:CarMeaRot17,DCC:LiuMes19,CC:TangLiu19}]\label{Prop:fam}
	
Let $m \in \N^*$ and $n=2^m$, the $n$-variable CMR function $f_n$, an $n$-variable LM function $g_n$ and an $n$-variable TL function $h_n$ have the following properties:
\begin{itemize}
	\item $\degg(f_n)=\frac{n}{2}$,
	\item $\degg(g_n)=n-1$,
	\item $\AI(h_n)=\frac{n}{2}$.	
\end{itemize}
	
\end{Prop}
\fi
\ifnum\short=1
% \vspace{-10pt}
% \subsubsection{Symmetric Functions.}
\else
\subsection{Symmetric Functions}%and Krawtchouk polynomials
\fi

The $n$-variable Boolean symmetric functions are those that are constant on each slice $\Ekn{k}{n}$ for $k\in [0,n]$. 
\ifnum\short=1
The set of $n$-variable symmetric functions is denoted $\SYM{n}$. Let $i\in [0,n]$, the \emph{elementary symmetric function} of degree $i$ in $n$ variables, denoted $\sigma_{i,n}$, is the function which ANF contains all monomials of degree $i$ and no monomials of other degrees; while, the \emph{indicator functions} of the slice of weight $k$ is the such that $\forall  x\in \mathbb{F}_2^{n}, \, \phikn{k}{n}(x) = 1 \text{ if and only if } \w(x) = k$.
\else
This class has been thoroughly studied in the context of cryptography, see \eg \cite{IEEE:Carlet04,IEEE:CanVid05,INDO:BraPre05,DM:SarMai07,IEEE:QFLW09,Latin:Meaux19,CCDS:Meaux21,IEEE:CarMea22}.
The set of $n$-variable symmetric functions is denoted $\SYM{n}$, and $|\SYM{n}|=2^{n+1}$. 
In this article we mainly consider two families of symmetric functions, which are both bases of the symmetric functions:

\begin{definition}[Elementary symmetric functions]\label{def:elemsymm}
Let $i\in [0,n]$, the elementary symmetric function of degree $i$ in $n$ variables, denoted $\sigma_{i,n}$, is the function which ANF contains all monomials of degree $i$ and no monomials of other degrees. 
\end{definition}

\begin{definition}[Slice indicator functions]\label{def:sliceind}
 Let $k\in [0,n]$, the indicator function of the slice of weight $k$ is defined as: 
 \ifnum\short=1
  $\forall  x\in \mathbb{F}_2^{n}, \, \phikn{k}{n}(x) = 1 \text{ if and only if } \w(x) = k$.
 \else
 \[\forall  x\in \mathbb{F}_2^{n}, \quad \phikn{k}{n}(x) = 1 \text{ if and only if } \w(x) = k.\]
  \fi
\end{definition}
%.
\fi




\ifnum\short=1
\else
\begin{Prop}[Properties of elementary symmetric functions]\label{prop:sigma}
	Let $n\in \N^*$, and $d\in [0,n]$:
	\begin{itemize}
		\item The function $\sigma_{d,n}$ takes the value $\binom{k}{d}\mod 2$ on the elements of the slice $\Ekn{k}{n}$.
		\item The function $\sigma_{2,n}$ takes the value $1$ only on the slices $\Ekn{k}{n}$ such that $k=2\mod 4$ or $k=3 \mod 4$. Moreover, for $n$ even, $\sigma_{2,n}$ is bent.
		
	\end{itemize}	
\end{Prop}

\begin{definition}[Threshold functions]\label{def:thr}
	For any positive integers $d \leq n+1$ we define the Boolean function $\thr{d}{n}$ as follows:
	\[ \forall x=(x_1,\dots,x_n) \in \F_2^{n}, \quad \thr{d}{n}(x)  = \left\{ 
	\begin{array}{l l}
	0 & \quad \text{ if } \w(x) <d, \\
	1 & \quad \text{ otherwise. }
	\end{array} \right.   \]  
\end{definition}


\begin{Prop}[Lucas' Theorem, \eg \cite{LucasTh}]\label{prop:Lucas}
	Let $a,b,p \in \N$ be integers such that $a>b$ and $p$ is a prime. Consider their $p$-adic expansions
	$a=\sum_{j=0}^q a_jp^j$ and $b=\sum_{j=0}^q b_jp^j$ such that $0\le a_j<p$ and $0\le b_j<p$ for each $j\in[0,q]$ and $a_q\ne 0$. Then
	\[\binom{a}{b} \equiv \prod_{j=0}^q \binom{a_j}{b_j} \pmod p.
	\]	
\end{Prop}

\begin{Prop}[Weightwise restricted Walsh transform and addition of symmetric function (~\cite{INDO:GinMea22}, Proposition $4$)]\label{prop:AddSym}
	Let $n\in \N^*$, $k\in [0,n]$ and $f \in \Bn{n}$, the following holds on $f+\phikn{k}{n}$
	\[\forall a\in \F_2^n, \forall i\in [0,n]\setminus \{k\}, \wtk{f+\phikn{k}{n}}{i}(a)=\wtk{f}{i}(a),\text{ and } \wtk{f+\phikn{k}{n}}{k}(a)=-\wtk{f}{i}(a).\]	
\end{Prop}

% We give two results relatively to Krawtchouk Polynomials we will use in the article. We refer to \eg ~\cite{book:MacSlo78} for more details on these polynomials and their properties.
% \begin{definition}[Krawtchouk Polynomials]\label{def:Kraw}
% 	The Krawtchouk polynomial of degree $k$, with $0\leq k\leq n$ is given by: $ \displaystyle \kraw{k}{\ell}{n}=\sum_{j=0}^{k} (-1)^j \binom{\ell}{j} \binom{n-\ell}{k-j}$. 
% \end{definition}
% 
% 
% \begin{Prop}[Krawtchouk polynomials relations]\label{prop:Kr}
% 	Let $n\in \N^*$ and $k\in [0,n]$, the following relations hold:
% 	\begin{enumerate}
% 		\item $\kraw{k}{\ell}{n} = \sum_{x \in \Ekn{k}{n}}
% 		(-1)^{a \cdot x}$, where $a \in \F_2^n$ and $\ell=\w(a)$,
% 		%\item $\kraw{n-k}{\ell}{n}=(-1)^\ell \kraw{k}{\ell}{n}$,
% 		%\item $\kraw{k}{n-\ell}{n}=(-1)^k \kraw{k}{\ell}{n}$,
% 		%	\item if $n$ is even and $k$ is odd, $\kraw{k}{n/2}{n}=0$,
% 		%		\item if $n$ is even, $\kraw{n/2}{1}{n}=0$.
% 		%	\item $\binom{n}{x}\kraw{k}{x}{n}=\binom{n}{k}\kraw{x}{k}{n}$.
% 		\item Proposition $5$~\cite{DCC:DalMaiSar06}
% 		For $n$ even, $k\in [0,n]$,
% 		\[\kraw{k}{n/2}{n}=\left\{ \begin{array}{l l}
% 		0 & \quad \text{ if } k \text{ is odd}, \\
% 		\displaystyle (-1)^{k/2} \binom{n/2}{k/2}& \quad \text{ if } k \text{ is even}.
% 		\end{array} \right.\]
% 	\end{enumerate}
% \end{Prop}




\ifnum\full=0
%%%%%%%%%%%%%%%%%%%%%%%%%%%%%%%%%%%%%%%%%%%%
%\bibliographystyle{splncs04}
\bibliographystyle{alpha}
\bibliography{abbrev3,add}
%%%%%%%%%%%%%%%%%%%%%%%%%%%%%%%%%%%%%%%%%%%%
\else
%%%%%%%%%%%%%%%%%%%%%%%%%%%%%%%%%%%%%%%%%%%%
\ifnum\short=1
\bibliographystyle{splncs04}
\bibliography{abbrev3,add}
\else
\bibliographystyle{alpha}
\bibliography{abbrev3,add}
\fi
%%%%%%%%%%%%%%%%%%%%%%%%%%%%%%%%%%%%%%%%%%%%
\fi

\end{document}
